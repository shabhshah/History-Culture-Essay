\documentclass[10pt]{article}

\usepackage{geometry}
\geometry{
	letterpaper,
	left=1in,
	right=1in,
	top=1in,
	bottom=1in,
	}

\usepackage{setspace}

\usepackage{textcomp}

\PassOptionsToPackage{hyphens}{url}\usepackage{hyperref}

\usepackage{gensymb}

\usepackage{hanging}

\setlength{\parindent}{15pt}

\begin{document}

\title{
	History Culture Essay\\
	Amcult 263 Section 004
	}

\author{
	Rishabh Shah\\
	rishabas\\
	4655 4192\\}

\date{March 31, 2017}

\maketitle
\thispagestyle{empty}

\newpage
\clearpage
\pagenumbering{arabic}
\doublespacing

After listening to ``Born on the Bayou'' by Creedence Clearwater Revival (CCR), one would be led to believe that the band and its members were born and raised in the backwoods of Louisiana.\footnote{Creedence Clearwater Revival, \textit{Bayou Country}, recorded in Hollywood, California, 1968, RCA Studios, 1969.} It would be hard to imagine otherwise, given the explicit imagery of the southern landscape and the iconic E7 chord that the song opens with.\footnote{Alex Abramovich, ``Where John Fogerty's Songs Come From,'' \textit{The New Yorker}, October 6, 2015, accessed March 20, 2017, \url{http://www.newyorker.com/culture/culture-desk/where-john-fogertys-songs-come-from}.} However, none of the members of CCR had ever been to Louisiana before John Fogerty wrote the track.\footnote{\textit{Ibid.}} In fact, the members of CCR were from Northern California, specifically El Cerrito, California, just east of San Francisco, far from the conventional idea of the American South.\footnote{Hank Bordowitz, \textit{Bad Moon Rising: The Unauthorized History of Creedence Clearwater Revival} (Chicago: Chicago Review Press, Incorporated, 2007), 7.} Thus, the natural question arises --- can Creedence Clearwater Revival be considered a Southern band given that the members are not from the South and have never physically been to the South? The answer is in fact yes, CCR can be considered a Southern band.

CCR was officially formed in December 1967 when they introduced themselves at a gig on Christmas Eve as ``Creedence Clearwater Revival'', however, Stu Cook, Doug Clifford, and brothers John and Tom Fogerty had been playing together for the last eight years.\footnote{\textit{Ibid.}, 40.} Younger brother John Fogerty met Stu Cook and Doug Clifford in junior high, and the trio stuck together until CCR's eventual collapse in 1972.\footnote{\textit{Ibid.}, 12.} After playing together for many years, they started backing older brother Tom Fogerty at many gigs around the El Cerrito area. Soon, the Blue Velvets (John, Stu, and Doug's trio) joined forces with Tom and started a band called the Golliwogs.\footnote{\textit{Ibid.}, 26.} The Golliwogs played at mostly at small venues while the musicians developed their style. John took over lead vocals, after being too shy to sing for a multitude of years; Stu switched to the bass guitar from the piano; and Tom took over role of lead rhythm guitarist.\footnote{\textit{Ibid.}, 27.} It was during this time that the Golliwogs would begin their transformation to becoming one of America's most popular band. According to Doug, it took eight years for them to ``make it'' simply because when they ``started, [they] were terrible.''\footnote{\textit{Ibid.}, 30.} Then, in 1966 both John and Doug were drafted as part of the Vietnam War effort.\footnote{\textit{Ibid.}} It would be this experience which would lead John to write the ninety-ninth greatest song of all time, ``Fortunate Son.''\footnote{``500 Greatest Songs of All Time,'' \textit{Rolling Stone}, April 7, 2011, accessed March 29, 2017, \url{http://www.rollingstone.com/music/lists/the-500-greatest-songs-of-all-time-20110407/creedence-clearwater-revival-fortunate-son-20110526}.} After a six month hiatus, the Golliwogs returned in full force. After Saul Zaentz bought Fantasy Records in 1967, he offered to sign the Golliwogs on one condition: they changed their name.\footnote{Bordowitz, \textit{Bad Moon Rising}, 39.} Many names were suggested, but the group decided on Creedence Clearwater Revival. It did not take long for the group to find a new weekly gig and a new identity, and then release their first album, the eponymous \textit{Creedence Clearwater Revival} which included covers such as ``Susie Q'' and ``I Put a Spell On You.''\footnote{Creedence Clearwater Revival, \textit{Creedence Clearwater Revival}, recorded in San Francisco, California, 1967-1968, Coast Recorders, 1968.} For the next four years, CCR would become one of the country's most popular bands, defining themselves as a rock and roll band influenced heavily by the American South.


\vfill
Word Count (per \url{https://wordcounter.net/}): 508

\newpage
\singlespacing
\begin{center}
\textit{{\large Bibliography}}
\end{center}

\begin{hangparas}{15pt}{1}
``500 Greatest Songs of All Time.'' \textit{Rolling Stone}. April 7, 2011. accessed March 29, 2017. \url{http://www.rollingstone.com/music/lists/the-500-greatest-songs-of-all-time-20110407/creedence-clearwater-revival-fortunate-son-20110526}. \\

Abramovich, Alex. ``Where John Fogerty's Songs Come From.'' \textit{The New Yorker}. October 6, 2015. accessed March 20, 2017. \url{http://www.newyorker.com/culture/culture-desk/where-john-fogertys-songs-come-from}. \\

Bordowitz, Hank. \textit{Bad Moon Rising: The Unauthorized History of Creedence Clearwater Revival}. (Chicago: Chicago Review Press, Incorporated, 2007). \\

Creedence Clearwater Revival. \textit{Bayou Country}. recorded in Hollywood, California. 1968. RCA Studios. 1969. \\

Creedence Clearwater Revival. \textit{Creedence Clearwater Revival}. recorded in San Francisco, California. 1967-1968. Coast Recorders. 1968. \\
\end{hangparas}

\end{document}