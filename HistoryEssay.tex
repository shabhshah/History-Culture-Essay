\documentclass[10pt]{article}

\usepackage{geometry}
\geometry{
	letterpaper,
	left=1in,
	right=1in,
	top=1in,
	bottom=1in,
	}

\usepackage{setspace}

\usepackage{textcomp}

\usepackage{hyperref}

\usepackage{gensymb}

\setlength{\parindent}{15pt}

\begin{document}

\title{
	History Culture Essay\\
	Amcult 263 Section 004
	}

\author{
	Rishabh Shah\\
	rishabas\\
	4655 4192\\}

\date{March 31, 2017}

\maketitle
\thispagestyle{empty}

\newpage
\clearpage
\pagenumbering{arabic}
\doublespacing

After listening to ``Born on the Bayou'' by Creedence Clearwater Revival (CCR), one would be led to believe that the band and its members were born and raised in the backwoods of Louisiana.\footnote{Creedence Clearwater Revival, \textit{Bayou Country}, recorded in Hollywood, California, 1968, RCA Studios, 1969.} It would be hard to imagine otherwise, given the explicit imagery of the southern landscape and the iconic E7 chord that the song opens with.\footnote{Alex Abramovich, ``Where John Fogerty's Songs Come From,'' \textit{The New Yorker}, October 6, 2015, accessed March 20, 2017, \url{http://www.newyorker.com/culture/culture-desk/where-john-fogertys-songs-come-from}.} However, none of the members of CCR had ever been to Louisiana before John Fogerty wrote the track.\footnote{\textit{Ibid.}} In fact, the members of CCR were from Northern California, specifically El Cerrito, California, just east of San Francisco, far from the conventional idea of the American South.\footnote{Hank Bordowitz, \textit{Bad Moon Rising: The Unauthorized History of Creedence Clearwater Revival} (Chicago: Chicago Review Press, Incorporated, 2007), 7.} Thus, the natural question arises --- can CCR be considered a Southern band given that the members are not from the South and have never physically been to the South? The answer is in fact yes, CCR can be considered a Southern band.

\newpage
\begin{center}
\textit{Bibliography}
\end{center}



\end{document}